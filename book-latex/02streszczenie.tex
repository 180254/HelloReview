% streszczenie
\begin{center}
\addcontentsline{toc}{chapter}{\numberline{}Streszczenie}
\textbf{Streszczenie}
\end{center}

W ramach niniejszej pracy inżynierskiej autor opracował system do wzajemnej oceny kodu programistycznego. System ten nie jest kolejnym typowym systemem wspomagającym słownikowo rozumiany przegląd koleżeński. Zrealizowany system ma na celu dostosowanie wspomnianego przeglądu do realiów dydaktycznych.

Realizując specyficzne wymagania takie jak anonimowość przeglądu, czy nadzór osoby prowadzącej kurs (i~powierzenie jej roli zleceniodawcy), wspomagany przegląd oddalił się od definicji tytułowego przeglądu wzajemnego. Autor jest głęboko przekonany, iż realizacja przeglądów prac podczas nauki, nie tylko przez prowadzącego kurs, ale także przez kolegów pozytywnie wpłynie na skuteczność usuwania defektów i~postęp nauki. System ten ma to umożliwić.

System został zrealizowany jako aplikacja webowa. Jego budowanie przyświecała idea wielokrotnego użycia. W~tym celu system głęboko integruje się z~powszechnie wykorzystywanym systemem kontroli wersji Git, oraz usługą GitHub. Prowadzący kurs nie tylko nie musi rezygnować z~wykorzystywanych już narzędzi, ale także bezproblemowo może poszerzyć warsztat narzędzi dydaktycznych.

\vspace*{\baselineskip}

\noindent\textbf{Słowa kluczowe:} \textit{przegląd koleżeński, wielokrotne użycie}

\vspace*{1.5\baselineskip}

\begin{center}
\addcontentsline{toc}{chapter}{\numberline{}Abstract}
\textbf{Abstract}
\end{center}

Within this thesis the author has developed and implemented a~system for peer review of~application source code. This is not another system that just supporting peer review as~understood in~the~dictionary. System has been adapted to~the~realities of~learning process in~schools.

By~carrying out specific requirements such as \textquote{review must be anonymous} or \textquote{review is supervised by~the~person leading the course}, assisted review move away from the definition. The author is deeply convinced that performing peer review while learning, not only by course leader, but also by colleagues positively affect the efficiency of removal of defects and will speed up learning courve. This system is to create opportunities.

The system has been implemented as~a~Web application. Software reusable is~very important in that system. It deeply integrates with one of~commonly used version control system - Git and a~GitHub service. Course leader can easily start using this system and does not have to give up already using tools.

\vspace*{\baselineskip}

\noindent\textbf{Keywords:} \textit{peer review, reusability}


% ex: set tabstop=4 shiftwidth=4 softtabstop=4 noexpandtab fileformat=unix filetype=tex spelllang=pl,en spell:
