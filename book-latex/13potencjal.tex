\chapter{Potencjał do badań naukowych}
Wymagania postawione przed systemem, w~celu zrealizowania określonego celu biznesowego, jasno ukazują, że opracowany system odbiega od definicji \textquote{przeglądu nieformalnego}, jakim jest tytułowy \textquote{przegląd koleżeński}. Odstępstwa, takie jak wprowadzenia procedury oceny, czy wymuszenie tej oceny przez wydanie polecenia osoby nadrzędnej, skłaniają ku klasyfikowaniu go jako "przeglądu formalnego". Trudno jednak wskazać bez wątpliwości jego klasyfikację. System ten jest swego rodzaju hybrydą rozwiązań opisanych już w~książkach, i~zbadanych pod kątem skuteczności.

\medskip
Wniosek ten zachęca do przeprowadzania badań nad skutecznością takiego mieszanego rozwiązania. Nie tylko w~kwestii skuteczności wykrywania defektów, ale także w~kontekście zyskiwanych umiejętności. Warto sprawdzić, czy istnieje zależność pomiędzy tym, że kursanci regularnie oceniali w~ten sposób prace kolegów, a~ich wynikami w~kursie. Potencjalnym zyskiem może być też spłaszczenie krzywej nauki, szybsze nabieranie właściwych nawyków czy staranniejsze wykonywanie zadań, z~uwagi na sam fakt bycia ocenianym nie tylko przez prowadzącego kurs, ale także przez kolegów.

\medskip
Przeprowadzanie tego typu badań wykracza jednak poza zdefiniowany zakres pracy. Praca skupia się nad jak najlepszym zaprojektowaniem systemu, implementacją zgodną ze sztuką, oraz opisaniem zrealizowanego systemu.


% ex: set tabstop=4 shiftwidth=4 softtabstop=4 noexpandtab fileformat=unix filetype=tex spelllang=pl,en spell:
